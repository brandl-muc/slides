\documentclass{beamer}
 \setbeamercovered{transparent}
 \usetheme{Madrid}
 \usecolortheme{lily}
 \geometry{paperwidth=160mm,paperheight=90mm}
\usepackage{listings}
\usepackage{accsupp}
\newcommand*{\noaccsupp}[1]{\BeginAccSupp{ActualText={}}#1\EndAccSupp{}}

\definecolor{ForestGreen}{RGB}{60, 160, 49}

\lstdefinestyle{Common}
{
    numbers=left,
    numbersep=1em,
    numberstyle=\tiny\color{red}\noaccsupp,
    frame=single,
    framesep=\fboxsep,
    framerule=\fboxrule,
    rulecolor=\color{red},
    xleftmargin=\dimexpr\fboxsep+\fboxrule\relax,
    xrightmargin=\dimexpr\fboxsep+\fboxrule\relax,
    breaklines=true,
    tabsize=2,
    columns=flexible,
}

\lstdefinestyle{C++}
{
	style=Common,
    language={C++},
    basicstyle=\ttfamily,
    keywordstyle=\color{blue}\ttfamily,
    stringstyle=\color{red}\ttfamily,
    commentstyle=\color{ForestGreen}\ttfamily,
    morecomment=[l][\color{gray}]{\#},
    backgroundcolor=\color{orange!10},
}

\lstdefinestyle{TinyC++}
{
	style=Common,
    language={C++},
    basicstyle=\ttfamily\tiny,
    keywordstyle=\color{blue}\ttfamily,
    stringstyle=\color{red}\ttfamily,
    commentstyle=\color{ForestGreen}\ttfamily,
    morecomment=[l][\color{gray}]{\#},
    backgroundcolor=\color{orange!10},
}


\lstnewenvironment{C++}{\lstset{style=C++}}{}
\lstnewenvironment{TinyC++}{\lstset{style=TinyC++}}{}

\title{An application for private inheritance?}
\subtitle{Lightning Talk for MUC++}	
\author{Matth\"aus Brandl}
\institute{EOS GmbH}
\date{2018-05-17}
\subject{Computer Science}

\def\code#1{\texttt{#1}}
\def\cite#1{\textit{\textcolor{blue}{#1}}}

\begin{document}

%%%%%%%%%%%%%%%%%%%%%%%%%%%%%%%%%%%%%%%%%%%%%%%%%%%%%%%%%%%%%%%%%%%%%%%

\frame{\titlepage}

%%%%%%%%%%%%%%%%%%%%%%%%%%%%%%%%%%%%%%%%%%%%%%%%%%%%%%%%%%%%%%%%%%%%%%%

\begin{frame}[fragile]
\frametitle{Wait, what?}
\framesubtitle{What is private inheritance}

\begin{C++}
class Derived : private Base
{};
\end{C++}

\begin{itemize}
\item all public and protected members of \code{Base} are accessible as private members of the derived class
\item private members of the base are never accessible unless friended
\item instances of \code{Derived} cannot be cast to \code{Base} as this relationship is inaccessible outside of \code{Derived}
\end{itemize}

Consequently this models HAS-A instead of IS-A.\\
But \cite{favor composition over inheritance} is also valid here, so HAS-A is better modelled by using a member variable because this causes less coupling. Also one is less likely to violate the Liskov Substitution Principle.
\end{frame}

%%%%%%%%%%%%%%%%%%%%%%%%%%%%%%%%%%%%%%%%%%%%%%%%%%%%%%%%%%%%%%%%%%%%%%%

\begin{frame}[fragile]
\frametitle{Exceptions to the rule}
%\framesubtitle{To every rule there is an exception}
Private inheritance should be used if one does not want to model IS-A but
\begin{itemize}
\item needs to override a virtual function
\item needs access to a protected member
\item needs to create an object before / destroy it after another base
\item needs to share a virtual base or needs to control the construction of a virtual base
\item wants to make use of the Empty Base Optimization, e.g. with policy-based design
\end{itemize}

Also see the \href{https://isocpp.org/wiki/faq/private-inheritance}{\beamergotobutton{C++ FAQ}} or \href{https://en.cppreference.com/w/cpp/language/derived_class%23Private_inheritance}{\beamergotobutton{cppreference.com}}.
\end{frame}

%%%%%%%%%%%%%%%%%%%%%%%%%%%%%%%%%%%%%%%%%%%%%%%%%%%%%%%%%%%%%%%%%%%%%%%

\begin{frame}[fragile]
\frametitle{The problem}
%\framesubtitle{Subtitle}
Suppose the following C API:

\begin{C++}
typedef struct
{
    /* raw owning pointer, it's C after all */
    char const * name;

    /* more variables that need resources ... */
} Widget;

int createWidget(Widget const ** widget);

void freeWidget(Widget const * widget);
\end{C++}

Your job now is to implement this API using C++.
\end{frame}

%%%%%%%%%%%%%%%%%%%%%%%%%%%%%%%%%%%%%%%%%%%%%%%%%%%%%%%%%%%%%%%%%%%%%%%

\end{document}
