\documentclass{beamer}
 \setbeamercovered{transparent}
 \usetheme{Madrid}
 \usecolortheme{lily}
\usepackage{listings}

\title{An application for private inheritance?}
\author{Matth\"aus Brandl}
\institute{EOS GmbH}
\date{2018-05-17}
\subject{Computer Science}

\definecolor{ForestGreen}{RGB}{60, 160, 49}
\def\code#1{\texttt{#1}}
\def\cite#1{\textit{\textcolor{blue}{#1}}}

\begin{document}

\frame{\titlepage}

\begin{frame}[fragile]
\frametitle{Wait, what?}
\framesubtitle{What is private inheritance}

\lstset{language=C++,
                basicstyle=\ttfamily,
                keywordstyle=\color{blue}\ttfamily,
                stringstyle=\color{red}\ttfamily,
                commentstyle=\color{ForestGreen}\ttfamily,
                morecomment=[l][\color{gray}]{\#}
}
\begin{lstlisting}
class Derived : private Base
{};
\end{lstlisting}

\begin{itemize}
\item all public and protected members of \code{Base} are accessible as private members of the derived class
\item private members of the base are never accessible unless friended
\item instances of \code{Derived} cannot be cast to \code{Base} as this relationship is inaccessible outside of \code{Derived}
\end{itemize}

Consequently this models HAS-A instead of IS-A.\\
But \cite{favor composition over inheritance} is also valid here, so HAS-A is better modelled by using a member variable because this causes less coupling. Also one is less likely to violate the Liskov Substitution Principle.

\end{frame}

\begin{frame}[fragile]
\frametitle{Exceptions to the rule}
%\framesubtitle{To every rule there is an exception}
Private inheritance should be used if one does not want to model IS-A but
\begin{itemize}
\item needs to override a virtual function
\item needs access to a protected member
\item needs to create an object before / destroy it after another base
\item needs to share a virtual base or needs to control the construction of a virtual base
\item wants to make use of the Empty Base Optimization, e.g. with policy-based design
\end{itemize}

Also see the \href{https://isocpp.org/wiki/faq/private-inheritance}{\beamergotobutton{C++ FAQ}} or \href{https://en.cppreference.com/w/cpp/language/derived_class%23Private_inheritance}{\beamergotobutton{cppreference.com}}.
\end{frame}

\end{document}
