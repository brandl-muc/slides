\documentclass{beamer}
 \usetheme{Madrid}
 \usecolortheme{lily}
 \geometry{paperwidth=160mm,paperheight=90mm}
 \beamertemplatenavigationsymbolsempty
\usepackage{listings}
\usepackage{accsupp}
\newcommand*{\noaccsupp}[1]{\BeginAccSupp{ActualText={}}#1\EndAccSupp{}}

\definecolor{ForestGreen}{RGB}{60, 160, 49}

\lstdefinestyle{Common}
{
    numbers=left,
    numbersep=1em,
    numberstyle=\tiny\color{orange}\noaccsupp,
    frame=single,
    framesep=\fboxsep,
    framerule=\fboxrule,
    rulecolor=\color{orange},
    backgroundcolor=\color{orange!10},
    xleftmargin=\dimexpr\fboxsep+\fboxrule\relax,
    xrightmargin=\dimexpr\fboxsep+\fboxrule\relax,
    breaklines=true,
    tabsize=2,
    columns=flexible,
}

\lstdefinestyle{C++}
{
	style=Common,
    language={C++},
    basicstyle=\ttfamily,
    keywordstyle=\color{blue}\ttfamily,
    stringstyle=\color{red}\ttfamily,
    commentstyle=\color{ForestGreen}\ttfamily,
    morecomment=[l][\color{gray}]{\#},
}

\lstdefinestyle{TinyC++}
{
	style=Common,
    language={C++},
    basicstyle=\ttfamily\tiny,
    keywordstyle=\color{blue}\ttfamily,
    stringstyle=\color{red}\ttfamily,
    commentstyle=\color{ForestGreen}\ttfamily,
    morecomment=[l][\color{gray}]{\#},
}


\lstnewenvironment{C++}{\lstset{style=C++}}{}
\lstnewenvironment{TinyC++}{\lstset{style=TinyC++}}{}

\title{An application for private inheritance?}
\subtitle{Lightning Talk for MUC++}	
\author{Matth\"aus Brandl}
\date{2018-05-17}
\subject{Computer Science}

\def\code#1{\texttt{#1}}
\def\titleinframe#1{{\usebeamercolor[fg]{structure} #1}}
\def\link#1#2{\href{#1}{\usebeamercolor[fg]{structure} \underline{#2}}}

\begin{document}

%%%%%%%%%%%%%%%%%%%%%%%%%%%%%%%%%%%%%%%%%%%%%%%%%%%%%%%%%%%%%%%%%%%%%%%

\frame{\titlepage}

%%%%%%%%%%%%%%%%%%%%%%%%%%%%%%%%%%%%%%%%%%%%%%%%%%%%%%%%%%%%%%%%%%%%%%%

\begin{frame}[fragile]
\frametitle{Wait, what?}
\framesubtitle{What is private inheritance}

\begin{C++}
class Derived : private Base
{};
\end{C++}

\begin{itemize}
\item all public and protected members of \code{Base} accessible as private members of \code{Derived}
\item private members of \code{Base} never accessible (unless friended)
\pause 
\item inheritance relationship not accessible outside of \code{Derived}, not \code{static\_cast}-able
\pause
\item models HAS-A instead of IS-A
\end{itemize}
\pause

However HAS-A is usually better modelled by using a member variable because this causes less coupling (\textit{favor composition over inheritance}).
\end{frame}

%%%%%%%%%%%%%%%%%%%%%%%%%%%%%%%%%%%%%%%%%%%%%%%%%%%%%%%%%%%%%%%%%%%%%%%

\begin{frame}[fragile]
\frametitle{Typical use cases for private inheritance}
Private inheritance should (amongst others) be used if one does not want to model IS-A but
\begin{itemize}
\item needs to override a virtual function
\pause
\item needs access to a protected member
\pause
\item wants to make use of the Empty Base Optimization (e.g. with policy-based design)
\end{itemize}
\pause
There are also other use cases, see the \link{https://isocpp.org/wiki/faq/private-inheritance}{C++ FAQ} or \link{https://en.cppreference.com/w/cpp/language/derived_class\#Private_inheritance}{cppreference.com}.
\end{frame}

%%%%%%%%%%%%%%%%%%%%%%%%%%%%%%%%%%%%%%%%%%%%%%%%%%%%%%%%%%%%%%%%%%%%%%%

\begin{frame}[fragile]
\frametitle{The problem}
%\framesubtitle{Subtitle}
Suppose the following C API:

\begin{C++}
typedef struct
{
    /* pointer to a resource, e.g., a C string */
    char const * foo;

    /* more variables that need resources ... */
    
} Widget;

int createWidget(Widget const ** widget);

void freeWidget(Widget const * widget);
\end{C++}

We want to implement this API using C++...
\end{frame}

%%%%%%%%%%%%%%%%%%%%%%%%%%%%%%%%%%%%%%%%%%%%%%%%%%%%%%%%%%%%%%%%%%%%%%%

\begin{frame}[fragile]
\frametitle{The C-ish approach}
\framesubtitle{Tedious but effective}
\begin{columns}
\begin{column}{0.45\textwidth}
\begin{TinyC++}
int createWidget(Widget const ** widget)
{
	Widget * newWidget =
			(Widget *) std::malloc(sizeof(Widget));
	if (!newWidget)
	{
		return OUT_OF_MEMORY;
	}
	*widget = {}; // zero initialize
    
	std::string	foo = frobnicate(/* ... */);
	if (foo.empty())
	{
		freeWidget(newWidget);
		return FROBNICATE_FAILED;
	}
	newWidget->foo = strdup(foo.c_str());
    
	/* ... */
    
	*widget = newWidget;
	return SUCCESS;
}
\end{TinyC++}
\end{column}
\begin{column}{0.45\textwidth}
\begin{TinyC++}
void freeWidget(Widget const * widget)
{
	std::free(widget->foo);
    
	/* ... */
    
	std::free(widget);
}
\end{TinyC++}
\end{column}
\end{columns}
\end{frame}

%%%%%%%%%%%%%%%%%%%%%%%%%%%%%%%%%%%%%%%%%%%%%%%%%%%%%%%%%%%%%%%%%%%%%%%

\begin{frame}[fragile]
\frametitle{The C-ish approach}
\framesubtitle{\code{malloc()} \& \code{free()} galore}
\begin{columns}[T]
\begin{column}{0.45\textwidth}
	\titleinframe{Advantages:}
	\pause
	\begin{itemize}
	\item straightforward to implement
	\item easy to understand
	\item low complexity
	\end{itemize}
\end{column}
\pause
\begin{column}{0.45\textwidth}
	\titleinframe{Disadvantages:}
	\pause
	\begin{itemize}
	\item manual ressource management
	\item raw owning pointers
	\item error prone
	\item hard to get right
	\item tedious
	\end{itemize}
\end{column}
\end{columns}
\end{frame}

%%%%%%%%%%%%%%%%%%%%%%%%%%%%%%%%%%%%%%%%%%%%%%%%%%%%%%%%%%%%%%%%%%%%%%%

\begin{frame}[fragile]
\frametitle{Automating ressource management}
\framesubtitle{Using the power of C++}
\begin{columns}
\begin{column}{0.45\textwidth}
\pause
\begin{TinyC++}
class ResourcedWidget : public Widget
{
public:
	explicit ResourcedWidget()
        : Widget()	// Default initialization!
	{}

	void setFoo(std::string const & value)
	{
		m_foo = value;
		foo = m_foo.c_str();
	}

	// more setters...
	
private:
	std::string m_foo;
};
\end{TinyC++}
\end{column}
\end{columns}
\end{frame}

%%%%%%%%%%%%%%%%%%%%%%%%%%%%%%%%%%%%%%%%%%%%%%%%%%%%%%%%%%%%%%%%%%%%%%%

\begin{frame}[fragile]
\frametitle{Automating ressource management}
\framesubtitle{Using the power of C++}
\begin{columns}
\begin{column}{0.45\textwidth}
\begin{TinyC++}
int createWidget(Widget const ** widget)
{
	try
	{
		auto newWidget = std::make_unique<ResourcedWidget>();
    
		std::string	foo = frobnicate(/* ... */);
		if (foo.empty())
		{
			return FROBNICATE_FAILED;
		}
		newWidget->setFoo(foo);

		// more setters...
    
		*widget = newWidget.release();
		return SUCCESS;
	}
	catch (std::bad_alloc const &)
	{
		return OUT_OF_MEMORY;
	}
}
\end{TinyC++}
\end{column}
\pause
\begin{column}{0.45\textwidth}
\begin{TinyC++}
void freeWidget(Widget const * widget)
{
	delete static_cast<ResourcedWidget const *>(widget);
}
\end{TinyC++}
\end{column}
\end{columns}
\end{frame}

%%%%%%%%%%%%%%%%%%%%%%%%%%%%%%%%%%%%%%%%%%%%%%%%%%%%%%%%%%%%%%%%%%%%%%%

\begin{frame}[fragile]
\frametitle{Automating ressource management}
%\framesubtitle{That's why we like C++}
\begin{columns}[T]
\begin{column}{0.45\textwidth}
	\titleinframe{Advantages:}
	
	\medskip
	Easier usage:	
	\begin{itemize}
	\pause
	\item automated resource management
	\pause
	\item \code{Widget} members are default initialized
	\pause
	\item easier \code{createWidget()} implementation
	\pause
	\item easier \code{freeWidget()} implementation
	\pause
	\item feels like a C++ class
	\end{itemize}
\end{column}
\pause
\begin{column}{0.45\textwidth}
	\titleinframe{Disadvantages:}
	
	\medskip
	Potential for resource leaks:
	\begin{itemize}
	\pause
	\item \code{static\_cast} can be forgotten during deletion
	\pause
	\item implementers can still access \code{Widget} members and use them wrongly \\
	(e.g. assign raw owning pointers)
	\end{itemize}

\end{column}
\end{columns}
\end{frame}

%%%%%%%%%%%%%%%%%%%%%%%%%%%%%%%%%%%%%%%%%%%%%%%%%%%%%%%%%%%%%%%%%%%%%%%

\begin{frame}[fragile]
\frametitle{Enter private inheritance}
\framesubtitle{Making \code{Widget} members inaccessible}
\pause
\begin{columns}
\begin{column}{0.45\textwidth}
\begin{TinyC++}
class ResourcedWidget : private Widget // private!
{
	// as before...

public:	
	Widget const * toWidget() const
	{
		return static_cast<Widget const *>(this);
	}
	
	static void deleteWidget(Widget const * widget)
	{
		delete static_cast<ResourcedWidget const *>(widget);
	}
};
\end{TinyC++}
\end{column}
\pause
\begin{column}{0.45\textwidth}
\begin{TinyC++}
int createWidget(Widget const ** widget)
try
{
	// as before...
    
	*widget = newWidget->toWidget();
	newWidget.release();
	return SUCCESS;
}
catch /* as before */
\end{TinyC++}

\begin{TinyC++}
void freeWidget(Widget const * widget)
{
	ResourcedWidget::deleteWidget(widget);
}
\end{TinyC++}
\end{column}
\end{columns}
\end{frame}

%%%%%%%%%%%%%%%%%%%%%%%%%%%%%%%%%%%%%%%%%%%%%%%%%%%%%%%%%%%%%%%%%%%%%%%

\begin{frame}[fragile]
\frametitle{Enter private inheritance}
%\framesubtitle{Making \code{Wrapper} members inaccessible}
\begin{columns}[T]
\begin{column}{0.45\textwidth}
	\titleinframe{Advantages:}
	
	\medskip
	\begin{itemize}
	\item \code{Widget} members not public anymore in \code{ResourcedWidget} context
	\item Easy to use right, hard to use wrong
	\end{itemize}
\end{column}
\pause
\begin{column}{0.45\textwidth}
	\titleinframe{Disadvantages:}
	
	\medskip
	\begin{itemize}
	\item still possible to delete a \code{ResourcedWidget} via a pointer to \code{Widget} \\ (but easier to remember the function than the \code{static\_cast})
	\item increased complexity, two additional functions necessary
	\item uses private inheritance for an IS-A relationship
	\end{itemize}
\end{column}
\end{columns}
\end{frame}

%%%%%%%%%%%%%%%%%%%%%%%%%%%%%%%%%%%%%%%%%%%%%%%%%%%%%%%%%%%%%%%%%%%%%%%

\begin{frame}[fragile]
\frametitle{Alternatives?}
	\begin{itemize}
	\item Do not introduce private inheritance and trust in that no one will use the \code{Widget} wrongly
	\pause
	\item Use aggregation and pass the pointer to the member to the client\\
	But now shared state between \code{createWidget()} and \code{freeWidget()} is necessary to find the correct \code{ResourcedWidget} instance for the given \code{Widget} pointer
	\end{itemize}
	
	\pause
	\bigskip
	Please share your opinion and ideas
	
	\bigskip
	There is a \link{http://coliru.stacked-crooked.com/a/e68e29b6461b083e}{working example on Coliru}
\end{frame}

%%%%%%%%%%%%%%%%%%%%%%%%%%%%%%%%%%%%%%%%%%%%%%%%%%%%%%%%%%%%%%%%%%%%%%%

\end{document}
